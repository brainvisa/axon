\documentclass[a4paper]{article}

% for extended summary
%\usepackage[summary]{nnsp2e}
% final paper
\usepackage{nnsp2e}
\usepackage[latin1]{inputenc}
\usepackage[english]{babel}
\usepackage{amssymb}%enumerate,theorem,setspace
\usepackage{shortvrb,subeqnarray}
\usepackage{color}
\usepackage{array}

% Example definitions.
% --------------------
\input{alphabet}   
\input{abrege}	
\input{abrmath}  
%\input{old/abrege}	
%\input{old/abrmath}  
\input{beginend}   
\input{transdef}
\title{Updates in how to manage Process/Categories documentation under \brv}

\author{Yann Cointepas, \today}
%\author{Jan Larsen and Cyril Goutte\\
%        Department of Mathematical Modeling, Building 321\\
%        Technical University of Denmark, DK-2800 Lyngby, Denmark\\
%        Phone: +45 4525 3923,3921\\
%        Fax: +45 4587 2599\\
%        E-mail: jl,cg@imm.dtu.dk\\
%        Web: eivind.imm.dtu.dk}


\begin{document}
\maketitle

\section*{General information}

Finished processes/categories documentation new features. Here is a little summary of these features:
\ben 
\item I changed the style of processes documentation. There is no table
     anymore. I am ready to go back if too many people complain.
 
\item Process and categories documentation are now in  \texttt{.minf} format 
    (by default it is XML minf). Hence, the text can be organized in several
     parts. Each part can be a classical string or an \textbf{XHTML} text (\verb'<XHTML>'
     tag in XML version only).
 
     The main differences between \textbf{HTML} and \textbf{XHTML} are:
    \bit
     \item All tags must be closed:, \texttt{<br>} is forbidden. Use \texttt{<br/>} (or
        \texttt{<br></br>}) instead.
        
     \item Tags are case significant: \texttt{<a href="toto.html">toto</A>} is
        not valid.
     \item Attributes values must be doubly-quoted: use \texttt{<table width="100\%">}
        instead of \texttt{<table width=100\%>}

     \item The tags must be correctly nested and closed.
 
     The good point is that XML documentation files are now legible easily
     and can be edited with any text or XML editor (I like copy/paste, \eg quanta+).
 \eit

  \item A process documentation is still a \texttt{*.procdoc} file with the same name
     as the \texttt{*.py} file of the process (e.g. AnaDoAll.py \fleche AnaDoAll.procdoc).
 
  \item The graphical editor is still working. It will check that the \textbf{HTML} you
     write is valid \textbf{XHTML}. If it is not, the only choice you have is to use a regular string.
 
  \item Categories documentation are parsed only in the main \brv process
     directory (this would change in the future).\\
      They are in
     \texttt{category\_documentation.minf} file in the corresponding category directory.
     Each file contain one text property per language (to date, only 'en' and 
     'fr' have been filled).
 
  \item \textbf{XHTML} texts are parsed to produce the final \textbf{HTML} documentation (the real
     fromat is \textbf{XHTML} because most browsers support it). During this parsing
     special tags and links are detected and converted :
    \bit
     \item \texttt{<a href="bvprocess://TheProcess>} has been replaced by a link to the
        process documentation that can be reached using the \emph{process id} \textbf{TheProcess} (\ie
        the name of the Python file without extention). 

        Example in the 'fr' text property: 
        
        \cl{\texttt{<a href="bvprocess://AnaDoAll">Le c�l�bre pipeline</a>}}

        Example in the 'en' text property: 

        \cl{\texttt{<a href="bvprocess://AnaDoAll">The famous pipeline</a>}}
        
     \item \texttt{<a href="bvcategory://TheCategory>} has been replaced by a link to the
        \emph{category} documentation.  \textbf{TheCategory} must be a relative path to
        the category directory (all in english and in lower case as it
        appear in \brv). 
        
        
        \cl{Example:\texttt{<a href="bvcategory://data management/import/diffusion">diffusion</a>}}
        
     \item Note that \verb'<_t_>text</_t_>' is actually replaced by the \brv translation of 'text'.

        \cl{Example :\texttt{<a href="bvcategory://data management"><\_t\_>data management</\_t\_></a>'}}

\eit
\een
\section*{Additional points}

  We have also added a special link to images. \texttt{bvimage://} stands for the \texttt{<img>} tag in documentation. It allows to make reference to an image without knowing where the final \textbf{HTML} file will
  be. 
  
  \cl{Example : \texttt{<img src="bvimage://diffusion/dti/rgbev.jpg"/>}}
  
  
\edoc
